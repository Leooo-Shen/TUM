\part{Platform as a Service -- PaaS}
\begin{itemize}
	\item Definition: 
	\begin{itemize}
		\item client: deploy \textbf{customer-created applications}
		\item cloud provider: provides web interfaces, programming languages, libraries, services and tools 
	\end{itemize}
	\item Rights as customer:
	\begin{itemize}
		\item application deployment
		\item configuration settings application-hosting environment
	\end{itemize}
	\item No control:
	\begin{itemize}
		\item cloud infrastructure: storage, network, servers, operation systems
	\end{itemize}
\end{itemize}

\section{Platform as a Service -- Microsoft Azure}
\begin{itemize}
	\item Focus: integration with \textbf{on-premise IT} -- company oriented
	\item Services:
	\begin{itemize}
		\item Azure Web Apps
		\item Azure Service Fabrics: microservices
		\item Azure Functions
	\end{itemize}
	\item Azure access:
	\begin{itemize}
		\item \textbf{Tenant}: a Microsoft Active \textbf{Directory} for managing accounts, groups and permissions. (eg: a company)
		\item \textbf{Subscription}: a subscription is \textbf{associated with a unique tenant}. A \textbf{resource container} for the tenant.
		\begin{itemize}
			\item can be divided into \textbf{multiple resource groups}. (eg: different teams) Each has its \textbf{own resource allocation}.
			\item account (eg: an employee): has a \textbf{role} in subscription: eg: account administrator, service administrator 
		\end{itemize}
	\end{itemize}
\end{itemize}

\subsection{Azure Web Apps Service}
\begin{itemize}
	\item \textbf{development, deployment and management of web applications} without managing VMs
	\begin{itemize}
		\item different programming languages provided: development
		\item a \textbf{managed web environment}: deployment
		\item automatic load balancing: \textbf{scale in/out}
		\item VM sharing possible
	\end{itemize}
	\item App development:
	\begin{itemize}
		\item creation of a \textbf{deployment user}
		\item creation of a \textbf{resource group}: name, location, template
		\item creation of a \textbf{app service plan}: VM instance type, number of VM instances, scale count, subscription (resources), region, pricing tier.
		\item creation of a web app $\rightarrow$ development ready
	\end{itemize}
	\item App scaling: triggered by \textbf{automatic load balancing}, scales \textbf{according to app service plan}, apps will be \textbf{scaled together}. 
	\begin{itemize}
		\item manual scaling
		\item \textbf{automatic} scaling: according to defined metric/target/time period. 
	\end{itemize}
	\item App monitoring: 
	\begin{itemize}
		\item per app: 
		
		\begin{table}[H]
			\begin{center}
				\begin{tabular}{|l|p{7cm}|}
					\hline
					metrics  &definition \\ \hline
					Average response time        & time taken to serve requests (s)  \\ \hline
					Average memory working set     & average amount of memory used (MiB)    \\ \hline
					CPU time      & amount of CPU consumed (s)    \\ \hline
					Data In  & amount of incoming bandwidth (MiB)    \\ \hline
					Data Out  & amount of outgoing bandwidth (MiB)    \\ \hline
					Requests  & total number of requests regardless of HTTP status code    \\ \hline
				\end{tabular}
			\end{center}
		\end{table}
		
		\item per app service plan:
		\begin{table}[H]
			\begin{center}
				\begin{tabular}{|l|p{9cm}|}
					\hline
					metrics  &definition \\ \hline
					CPU percentage        & \textbf{average} CPU used \textbf{across all instances} (s)  \\ \hline
					memory percentage     & \textbf{average} memory used \textbf{across all instances} (MiB)    \\ \hline
					Data In  & \textbf{average} amount of incoming bandwidth \textbf{across all instances} (MiB)    \\ \hline
					Data Out  & \textbf{average} amount of outgoing bandwidth \textbf{across all instances} (MiB)    \\ \hline
					Disk Queue Length  & \textbf{average} amount of read \& write requests queued on storage (disk I/O)    \\ \hline
					HTTP Queue Length  & \textbf{average} amount of http-requests (plan load) \\ \hline
				\end{tabular}
			\end{center}
		\end{table}
		\item granularity (level of detail of data): minute/hour/day granularity metrics
	\end{itemize}
\end{itemize}


\subsection{Microservices Platform -- Azure Service Fabric}

\subsubsection{Monolithic VS. Microservices}
\begin{itemize}
	\item \textbf{Monolithic} service:
	\begin{itemize}
		\item tiered architecture (eg: presentation/user interface -- logic -- data layer)
		\item \textbf{replication} of the \textbf{whole app} to other instances
		\item all states stored in a single database.
		\item testing and failure: one part fails, the whole app fails. Changes of one part requires testing/deployment of whole app.
	\end{itemize}
	\item \textbf{Microservice}: 
	\begin{itemize}
		\item app is \textbf{decomposed} into microservices. 
		\item \textbf{individual replication} of each microservice over instances. \textbf{individual scaling}. 
		\item stateless services: \textbf{individual database} connected to microservice for state storage.
		\item stateful services: state is stored in the service. 
		\item testing and failure: \textbf{individual} testing and deployment. If one microservice fails, the rest is unaffected.
		\item increased network traffic and latency sensitivity: communication between microservices, especially chatty services.
	\end{itemize}
\end{itemize}

\subsubsection{Service Fabric}
\begin{itemize}
	\item platform for \textbf{microservice based applications}
	\item Transformation of a monolithic service to a microservice:
	\begin{itemize}
		\item lift and shift: containerize existing codes in service fabric
		\item modernization: add new microservices alongside the containerized codes
		\item innovate: break the monolithic service into microservices based on needs.
	\end{itemize} 
	\item Fabric cluster: \textbf{a set of virtual/real machines (nodes)} where microservices are deployed.
	\begin{itemize}
		\item service - node allocation: placement constraints of services, properties specified by node in key-value pairs.
		\item load balancing: \textbf{distributes/migrates} services to other nodes, \textbf{distributes requests}
		\item scaling of a whole cluster possible: app scaling
	\end{itemize}
	\item service models:
	\begin{itemize}
		\item reliable services model
		\item reliable actors model: \textbf{stateful} objects, actor \textbf{communicates} by receiving and sending \textbf{messages}.actor reacts faster than stateless services.
		\item guest executables: treated like stateless services
	\end{itemize}
	\item \textbf{service partition}: a group of service instances 
	\begin{itemize}
		\item stateless service: \textbf{not partitioned}. all instances belong to a \textbf{single partition}. eg: collection, gallery, overview
		\item stateful service: \textbf{frequently partitioned or replicated}. eg: by country, name, payment status  
	\end{itemize}
	\item testing:
	\begin{itemize}
		\item restart nodes/partitions of service instances
		\item simulate load balancing, failover, app-upgrade
		\item invoke data loss
		\item simulate scenarios: chaos(continuous overlapping faults), failover(continuous overlapping faults targeting single partition)
	\end{itemize}
\end{itemize}