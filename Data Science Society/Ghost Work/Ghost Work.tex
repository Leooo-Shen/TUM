\documentclass[12pt]{article}

\usepackage[english]{babel} % en
\usepackage{graphicx}
\usepackage{hyperref}
\usepackage[T1]{fontenc}
\usepackage[utf8]{inputenc}
\usepackage{setspace}
\usepackage[a4paper, total={6in, 8in}]{geometry}
\usepackage{parskip}
\usepackage{fancyhdr}
\usepackage{cite}
\usepackage{units}
\usepackage[htt]{hyphenat}
\usepackage{enumitem}
\usepackage{relsize}
\usepackage{listings}
\usepackage{lipsum}
\usepackage{titlesec}

\titleformat*{\section}{\large\bfseries}

\title{\textbf{Ghost Work: giving up on mental health for a source of income}  \\[0.4em] \smaller{} Final Paper for Data Science Society}
\author{Filippo Simonazzi - 03730725}
\date{Summer Semester 2020}
\pagestyle{fancy}

\begin{document}
	
	\maketitle
	
	\lfoot{}
	\cfoot{}
	\rfoot{\thepage}
	
	\newpage
	\setcounter{tocdepth}{2}
	\tableofcontents
	\newpage

	\section{Introduction}
	\textit{Ghost work} consists in the employment of cheap human labor to execute tasks which computers can not perform automatically. These include data labeling, tuning artificial intelligence results and cleaning datasets: in short, checking results of algorithms and filtering them.
	
	Social media and internet services providers heavily depend on it, since they aim to reduce inappropriate content and improve the user experience, hence most of them either yield their own team, mainly outsourcing tasks, or rely on big companies hiring ghost workers to offer services. 
	
	\section{Ghost work: what does it really imply?}
	This kind of work has pros and cons: it does not generally require specific knowledge or prerequisites, it can be done remotely from anywhere as long as the employee has an internet connection and it is not particularly time consuming. However, it merely consists of staring at a computer screen, clicking at images and sliders, while performing tasks which are not at all stimulating and give zero experience.
	
	Ghost work could therefore be considered as an easy side gig to earn some extra money without the opportunity of acquiring additional skills or raises and promotions.
	
	Nevertheless, especially with the crisis happening nowadays, a lot of people have no additional source of income and are forced to dedicate all of their time to this type of tasks. 
	
	This raises further issues: first of all, most workers are independent contractors and therefore have \textbf{no fixed hours}, thus work is acquired by constantly refreshing web pages checking whether new tasks have been published. 
	
	Employees are required to complete an amount of daily, weekly or monthly reviews, and are subject to a high level of pressure to achieve these numbers. Having a policy of first-come-first-served not only causes stress, but also competitiveness, risk of neglecting real-life activities and relationships, while contributing to a toxic environment.
	
	Secondly, since ghost work often pays a little more than minimum wage and can be done comfortably at home, it does not encourage seeking for new job opportunities. People aim to work full time for even years, while not learning any useful concept in the meantime.
	
	There is barely any interaction between colleagues, and no clear hierarchy in the company: this leads to having \textbf{no such concept as a workplace or a team}, completely eliminating the need of soft skills and consequently their development. 
	
	Therefore, this kind of work can be easily thought of demeaning, humiliating, a dead-end career without any possibility of improvement. Furthermore, the content contractors are dealing with is often \textbf{graphic or vulgar}: since their duty is to clean and check results, it is likely that some of them are immoral. 
	
	\section{Lack of ethics, immorality and mental health concerns}
	Some of the content provided by Twitter, YouTube and Facebook involves suicides, massacres and abuse, and prolonged exposition may lead to long-term psychological damage. 
	
	14 current and former moderators in Manila described a workplace where \textit{nightmares, paranoia and obsessive ruminations} were common consequences of the job. Several described seeing colleagues suffer mental breakdowns at their desks. One of them said he attempted suicide as a result of the trauma.
	
	Confiding in friends is forbidden because of the confidentiality agreements, it is tough to opt out of this kind of tasks, and that daily accuracy targets create pressure not to take breaks\cite{washington}.
	
	In addition, workers are often to understand foreign languages or cultures, and this places an enormous burden on them since results may have multiple meanings or be more inappropriate for some audience rather than another.
	
	They are subject to rules including being unable to leave the property during breaks, being banned from freely using their phones, even in emergencies, and having restrictions on breaks.\cite{bi}
	
	Furthermore, not having a 9-5 dynamic can lead to burnout, treating people like machines giving them repetitive duties without a maximum amount to be performed, which becomes dehumanizing in the long term. 
	
	In the meanwhile, companies profit off people in need of income, trying to improve their reputation after failing to adequately moderate misinformation and offensive posts. 
	
	\section{Political and social effects}
	With the increasing popularity of ghost work, awareness about the poor conditions of employees has also been raised. Since they are isolated and anonymous, performing simple and repetitive tasks while under non-disclosure agreements, \textit{ghost work is barely considered work at all}: it completely lacks standards and does not ensure well-being of contractors. 
	
	Multiple investigations have been conducted, resulting in shocking facts: moderators may earn less than 30.000\$ per year\cite{theverge}, resorting to drug abuse to cope with trauma, having their issues downplayed by counselors hired by their company.
	
	This has naturally had consequences among society: books have been published to raise awareness, websites offer information to the public, and the topic is often brought up in conferences. However, it can be affirmed that such content is only found by specifically searching for it, and most people are still oblivious to this niche.
	
	\textbf{Press and newspaper} have most likely had the biggest impact, exposing multinationals and interviewing workers to bring to light their conditions. Most articles brutally unveil the reality of ghost work, and might be part of the reasons why companies are finally taking steps.
	
	On the other hand, reviewing content is an occupation accessible to everyone, and the number of position in large companies is huge, sometimes more than 10.000\cite{washington}. This could be seen as a great way to employ people in need and give them some sort of income. 
	
	Workers who are not often exposed to inappropriate content mostly report positive experiences, having flexible hours and relatively simple tasks, making ghost work one of the most popular online jobs. They themselves might not be even aware of the awful conditions their “colleagues” are subject to.
	
	In short, it seems like the main social issue is the lack of aid in case of inappropriate content, rather than salaries or lack of personal growth.
	
	Balancing emotionally sensitive work while meeting metrics and performance goals, while providing breaks and emotional support\cite{washington} would be the ultimate objective to make ghost work ethical while not abandoning its purpose. 
	
	\section{Are companies taking responsibility?}
	Companies themselves admitted that positions of ghost work have been created in a hurry, \textbf{without any plan of psychological care} or assurance of a good workplace condition\cite{washington}. 
	
	Therefore, Facebook announced it would raise wages to \$22 an hour for its U.S. content moderators. Last year, Amazon raised its minimum wage as well to \$15 an hour after facing harsh criticism over poor pay\cite{bi}. However, there is no clear statement regarding work outside of the United States.
	
	There also have been actions to recommend guidelines for content, creating common standards to allow workers to take breaks, vent and get help from therapists --- whether those are actually being followed is currently unknown. 
	
	YouTube says moderators worldwide have “regular” access to counseling, and that they do not review content for more than five hours a day. But it declined to provide specific information.
	
	Outsourcing companies such as Appen have adopted new methods such as labeling tasks as not safe for sensitive people, and giving the possibility to completely opt out of them. 
	
	Offering support and raising wages, or at least claiming to do so, seems to be the compromise companies are giving, since confidentiality agreements are not being reviewed. 
	
	\section{Potential additional steps to be taken}
	Since the purpose of ghost work is supplying with human labor to what algorithms ae unable to do, the question which comes naturally is: can machine learning techniques be improved to avoid this need? 
	
	Since this seems unlikely with the current computational power and state of research, the creation of \textbf{permanent positions} must be accompanied by rights and better conditions for employees. Each one of them should be individually valued and cared for, and mental health concerns must not be disregarded.
	
	The concept of workplace must evolve, providing opportunities to get raises and promotions or even better base salary. A maximum number of working hours should also be introduced \textbf{in all continents}, according to local laws. 
	
	Another potential solution would be encouraging corporations to rely on companies specialized in ghost work. It is safe to assume that almost every big tech firm relies on ghost work, and it is nearly impossible to have them all comply to the same standard. Having fewer companies to be accountable would simplify the process of unification of regulations.
	
	Furthermore, this way tasks would be diversified since different platform require different moderation, giving workers more space and less possibility of repetitively encountering offensive content. 
	
	Finally, workers should be encouraged to support and connect with one another, to share their experiences, enforce their rights as a group and come together to ask for protection. 
	
	\section{Conclusion}
	Ghost work is likely to be needed more and more in the future, since automation is possible only to an extent and usage of technology is quickly increasing among all fields. 
	
	Therefore, \textbf{an increase of the number of employees is to be expected}, yet this has to be accompanied with improvement of their conditions, standardization of salary and procedures to protect them from trauma and burnout. 
	
	Hopefully as the amount of ghost workers rises, society will also be more aware of the potential negative consequences and the poor workplace conditions, pressuring companies to take practical action.
	
	\clearpage
	\begin{thebibliography}{}
		\footnotesize
		
		\bibitem{bbc}
		\href{https://www.bbc.com/worklife/article/20190829-the-ghost-work-powering-tech-magic}{Edd Gent, Armies of workers help power the technological wizardry that is reshaping our lives – but they are invisible and their jobs are precarious, 2019}
		
		\bibitem{washington}
		\href{https://www.washingtonpost.com/technology/2019/07/25/social-media-companies-are-outsourcing-their-dirty-work-philippines-generation-workers-is-paying-price/}{Elizabeth Dwoskin, Jeanne Whalen and Regine Cabato, Content moderators at YouTube, Facebook and Twitter see the worst of the web — and suffer silently, 2019}
		
		\bibitem{theverge}
		\href{https://www.theverge.com/2019/2/25/18229714/cognizant-facebook-content-moderator-interviews-trauma-working-conditions-arizona}{Casey Newton, The Trauma Floor: The secret lives of Facebook moderators in America, 2019}
		
		\bibitem{bi}
		\href{https://www.businessinsider.com/facebook-moderators-complain-big-brother-rules-accenture-austin-2019-2?r=DE\&IR=T}{Rob Price, Facebook moderators are in revolt over 'inhumane' working conditions that they say erodes their 'sense of humanity', 2019}
		
		\bibitem{Glassdoor}
		\href{https://www.glassdoor.de/Bewertungen/Appen-Bewertungen-E667913.htm}{Source: Glassdoor}
		
	\end{thebibliography}
	
	
\end{document}